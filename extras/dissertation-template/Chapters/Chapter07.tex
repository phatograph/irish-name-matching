\chapter{Conclusion}
\label{ch:conclusion}

We successfully developed an extensible web service system to match names.
The system is initially encoded with 4 matching algorithms,
Levenshtein distance, soundex, Irish soundex, and lookup table.
We also present a web interface for a client to use the system
from the web browser.

The system is designed to be extended with simple inheritance, thus developer
can understand and develop further algorithm easily. In early state
simple design is enough to serve the purpose, so we follow the
\emph{Kiss principle} \cite[]{kiss}.

However, we have encountered some problem, also there are still many rooms
for future works. We will describe these in following sections.

\section{Encountered problem}

The major problem is that the current system takes too long to process
and also use too much memory. It has not been properly optimised
in term of performance. These following techniques might improve
our system furthermore.

\subsection{Memoization}

Memoization is the process of storing a computed value to avoid
duplicated work by repetitive calls. While each algorithm calculats
\emph{similarity score}, there might be many repetitive calculations
or database queries.

Ruby has a conditional assignment operator \texttt{\textbar \textbar =} \cite[]{cao}
which is commonly used for memoization. By doing so, it can improve
performance of the system and reduce the number of database calls \cite[]{memoization},
thus shorten response time and lower memory usage.

\subsection{find\_in\_batches}
Matching large amount of name causes high memory consumption and may lead to
out of memory situlation, especially in environment which memory are crucial
and expensive such as remote server.

Rails provides \texttt{find\_in\_batches} which operates an array in batches,
thus greatly reducing memory consumption \cite[]{fib}. We can apply
the same principle to our \emph{base name} and \emph{to-match name},
also to the \emph{controller} (section \ref{sec:mvc}).

\subsection{Replace RDBMS with NoSQL}

Current database system (section \ref{sec:testenv}) is a
\emph{Relational database management system (RDBMS)} \cite[]{rdbms} and the system
relies on traditional database queries. By replacing this with high speed
NoSQL database such as Redis \cite[]{redis}, which is one of the fastest
NoSQL \cite[]{redis2}, we can obtain better performance while using
lookup table algorithm.

\section{Future works}

\subsection{More soundexes}

\subsection{Lookup table for other languagues}

\subsection{Improve web interface result}

% overwhelm
