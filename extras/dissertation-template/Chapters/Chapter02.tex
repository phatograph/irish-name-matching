\chapter{Related Work}
\label{ch:relatedwork}

From research questions on section \ref{sec:rq}, there are three aforementioned
terms that will be a \emph{core} research fields of this project.
These fields are \emph{name matching}, \emph{web service},
and \emph{extensible platform}.

\section{Name matching}

There are many methods for matching names. This project encodes
various of them at the starting state.

\subsection{Edit distance}

\begin{quotation} \noindent
Edit distance is a way of quantifying how dissimilar two strings
(e.g., words) are to one another by counting the minimum number
of operations required to transform one string into the other.

-- Edit distance, \citet{editdistance}
\end{quotation}

Old-fashioned string operation way of comparing two string
could work with name matching too. Levenshtein distance \citep{levenshteindistance}
is chosen to implemented in this project.

% http://en.wikipedia.org/wiki/Edit_distance
% http://en.wikipedia.org/wiki/Levenshtein_distance
% http://www.levenshtein.net/index.html
% http://www.csse.monash.edu.au/~lloyd/tildeAlgDS/Dynamic/Edit/

\subsection{Soundex}

% This project relies on matching names of individuals from documents
% created in the early 19th century, a time when spelling was not standardized,
% handwriting could be difficult to read, contractions were used
% and many people were illiterate and so dictated their names to
% whoever was writing them down. This is a common problem in genealogy
% and various solutions have been found.

Soundex \cite{soundex} encodes a name (or any string) into a 4 character code
which represents an essence of its sound as pronounced in English.
The idea is to encode letters with similar sound into the same group,
and ignore vowels (unless it is the first letter).
For example, \emph{Smith} is translated to \texttt{S530}, and
\emph{Simon} is translated to \texttt{S550}.

% ‘Traditional Soundex’
% is a way of encoding a name for indexing, searching and matching
% that translates the name into a 4 character code that tries
% to characterise the essence of its sound in speech.

% The motivation is that similar sounding names, spelled differently,
% will produce the same code and so can be found when searching
% for the name in indexes. Soundex coding works on the assumption
% that consonant sounds remain more immutable in speech than vowels
% and that the first letter of a name rarely is mistaken.

% Therefore, the code consists of the first letter of the name
% followed by three digits encoding the first three consonant sounds
% in the name in the range 1..6, the letters being grouped
% into generally similar sounding categories (for example,
% both ‘m’ and ‘n’ share the code ‘5’). The codes for long names
% are truncated while short names are padded out with zeroes
% so all Soundex codes are 4 characters long.

% This makes for efficient searching for names in computer systems,
% results being based on an approximation of the sound of a name
% and so returning matches that traditional alphabetical searches
% would ignore.

% However, the Soundex algorithm can encode
% quite different names the same (for example McDonald and McAdam)
% and so produce spurious results. It also relies on the immutability
% of the first letter, so eliminating possible identically sounding names
% (for example Kane and Caine). Other techniques have been used
% to overcome these disadvantages as well as the western assumptions
% behind Soundex. These include Fuzzy Matching based on the Levenshtein distance
% formula, the Daitch-Mokotoff Soundex System (for Yiddish)
% and Beider-Morse Phonetic Matching (BMPM).

\subsection{Lookup Table}

% Irish surnames are a mixture of various influences reflecting
% the country’s history. In 1841, the majority used on official documents
% were either Anglicisations or translations of original Celtic names,
% using various spellings.

% Matheson, the assistant registrar-general in Dublin,
% used the civil registers in 1894 to produce a survey of names
% classified by original language, distribution and frequency.
% In 1901, he developed this into a full classification system with
% the aim of helping in searches of the register indexes.

% He illustrates it with examples collected from registry offices,
% where many of members of close families used different forms of their surnames
% (for example, a man called Smith registered as dead by his son using
% the name O’Gowan). He used this information to classify the surnames of Ireland
% into 2091 groups of alternate forms of the same name. For example,
% group 1897 consists of Smith, Smyth, Smythe, Smeeth, Gowan, O’Gowan,
% MacGowan, McGowan, M’Gowan, Goan, Going, Gow, Magough.

% However, his classification is far from complete and also includes
% multiple mapping between names (that is, the same name can occur
% in more than one group).

\section{Web service}

\section{Extensible framework}
