\chapter{Introduction}
\label{ch:introduction}

This project contains some backgrounds which are not part of computer science.
Here introduces adequate information in order to help getting start.

\section{Motivation}

The first civil registration in Ireland was performed on 1864 \cite[]{irishregistration}.
Before that time census meterials were mostly lost or incomplete. So genealogical researches
need to rely on parish records and also some `census substitute' documents,
such as land ownership and tenancy records\footnote{\cite[]{adamw} section 1.1}.

However, for these documents, each of them usually does not contain enough
information to indentify individuals. Some of them contains name and address,
whereas others might contain only name. In order to fulfil missing information
of one individual that scattered among many documents, \emph{Record linkage}
is one method to do so.

\graffito{``Record linkage is used in historical research, social studies,
marketing, administration and government as well as in genealogy'' \\
-- \citet[]{adamw} section 2.2}

Record linkage uses a person\textquotesingle s name as a basis to link that
person\textquotesingle s information between many documents.
Together with other coherant attributes to ensure the link is correct, a
more complete information about that person can be obtained.

In addition, this is not only just for one person. We can assume the relationship
of the person to others that might be close to, and apply the information
to those people as well. For example, if we know that there is a record
that is believed to consist of people from the same area in each page \cite[]{morpeth}
(but no area or address is mentioned, or some is missing in the page). And we can find
one or more person\textquotesingle s addresses in that page by using record linkage.
We might be able to apply those addresses to all people in that page as well.

Apparently linking or matching person\textquotesingle s name is important in the process.
Unfortunately, in the 19\textsuperscript{th} century, in Ireland, there was no standard
of the spelling of names, handwriting could be difficult to read
and contractions or abbreviations were often used. Many people were not literate,
so they asked literate people to write their names.
This way even names with the same pronounciation and for the same individual
could be written in many different ways, depending on who wrote them.

In addition to the various ways of spelling one\textquotesingle s name,
people from this time also often use Irish names which equivalent to modern names,
for example, Irish version of \emph{Smith} could be \emph{Gowan}.
There are also some Irish prefixes like \emph{O\textquotesingle}, \emph{M\textquotesingle}, \emph{Mac},
etc. When combined together this would result in \emph{O\textquotesingle Gowan} or
\emph{M\textquotesingle Gowen}, and so on.

An example list of possible equivalent Irish names of \emph{Smith}
could be as follow.

\begin{quotation} \noindent
Smith, Smyth, Smythe, Smeeth, Going, Gowing, Maizurn, McGhoon, MaGough,
M\textquotesingle Ghoon, MacGivney, MacGivena, M\textquotesingle Givena,
MacGhoon, M\textquotesingle Evinie, McGivney, MacEvinie, McGivena,
M\textquotesingle Givney, McEvinie, MacAvinue, M\textquotesingle Avinue,
McAvinue, McCona, MaGowen, MaGowan, MaGovern, MaGeown, McGowan, McGoween,
McGown, M\textquotesingle Cona, MeCowan, MeGowan, MacGown, MacGoween,
MacGowan, MacCona, M\textquotesingle Gowan, M\textquotesingle Gowen,
M\textquotesingle Gown, Ogowan, O\textquotesingle Gowan, Gowen,
Gowan, Gow, Goan
\end{quotation}

At present time, when historical researchers try to trace people back
using historical records, they would encounter this problem of
name variations.

Various solutions have been created to find
matching different names that refer to the same person. However,
for our extent knowledge, there is yet no public system which encodes
those solutions together and provides a service of name matching.
This project is to create one system to archieve this.

\section{Research Questions}
\label{sec:rq}

From the motivation, we address our research questions as follow.

\begin{enumerate}
  \item Can we provide a web service to match names, where matching can be
    a complicated process because of the way people record their names.
  \item Can the web service act as a platform system for general names or words
    matching system so that it can be extended to other languages as well.
\end{enumerate}

The first question derives directly from the motivation.
The second question is an enhancement for the system. It can be designed
as a more general purpose matching system rather than just specified
only for Irish names. Therefore it should be extensible for any further
matching methods to be developed in the future.

In addition to the web service, web interface is to be introduced as well
for the purpose of user friendly usage, individual usage, and demonstration.

\section{Objective and Aims}

The objective of this project is to provide a web service that
encodes several of matching methods and produces matching
results between two list of names.

The project aims to be a part of a bigger system, such as
genealogy research. These client systems, at some point,
they might need a service of a name matching on demand, so then they can use this
web service, providing their lists of name, methods be be used,
and threshold as inputs, and get matching results for their further usage.

We would start by focusing on Irish \textit{surname} first.
For any further kind of names we would leave it for future works.

\section{Report Structure}

This report is separated into four parts, The Background, The Solution,
and Appendix.

\begin{description}
\item[The Background:]
  Current part, states about background of this project. Introduces
  the initial problem, also some historical situations and terms
  which are not resident to computer science. Also related works
  that are involved in the project.
\item[The Solution:]
  The implementation to solve the problem. Details about algorithm,
  tools, language, frameworks, etc. which being used in the project.
\item[The Conclusion:]
  The outcome of the project. Evaluation of its performance,
  encountered problems, and future works for extending and improvements.
\item[Appendix:]
  The `user manual' of the project. Presents technical aspects,
  for example, how to use the web service in real world situation,
  or how to create an environment to host this project.
\end{description}
