\chapter{Introduction}
\label{ch:introduction}

This project contains some backgrounds which are not part of computer science.
Here introduces adequate information in order to help getting start.

\section{Motivation}

In the 19\textsuperscript{th} century, in Ireland, there was no standard
of the spelling of names, handwriting could be difficult to read,
contractions or abbreviations were used. Many people was not lliterate,
so they asked or dictated literate people to write their names.
This way even names with the same pronounciation could be written
in many different ways, depending on who wrote them.

Apart from various ways of spelling one\textquotesingle s name,
people from the time use Irish names which equivalent to modern names,
for example, Irish version of \emph{Smith} could be \emph{Gowan}.
There are also some Irish prefixes like \emph{O\textquotesingle}, \emph{M\textquotesingle}, \emph{Mac},
etc. When combined together this would result in \emph{O\textquotesingle Gowan} or
\emph{M\textquotesingle Gowen}, and so on.

An example list of possible equivalent Irish names of \emph{Smith}
could be as follow.

\begin{quotation} \noindent
\emph{Smith},
\emph{Smyth},
\emph{Smythe},
\emph{Smeeth},
\emph{Going},
\emph{Gowing},
\emph{Maizurn},
\emph{McGhoon},
\emph{MaGough},
\emph{M\textquotesingle Ghoon},
\emph{MacGivney},
\emph{MacGivena},
\emph{M\textquotesingle Givena},
\emph{MacGhoon},
\emph{M\textquotesingle Evinie},
\emph{McGivney},
\emph{MacEvinie},
\emph{McGivena},
\emph{M\textquotesingle Givney},
\emph{McEvinie},
\emph{MacAvinue},
\emph{M\textquotesingle Avinue},
\emph{McAvinue},
\emph{McCona},
\emph{MaGowen},
\emph{MaGowan},
\emph{MaGovern},
\emph{MaGeown},
\emph{McGowan},
\emph{McGoween},
\emph{McGown},
\emph{M\textquotesingle Cona},
\emph{MeCowan},
\emph{MeGowan},
\emph{MacGown},
\emph{MacGoween},
\emph{MacGowan},
\emph{MacCona},
\emph{M\textquotesingle Gowan},
\emph{M\textquotesingle Gowen},
\emph{M\textquotesingle Gown},
\emph{Ogowan},
\emph{O\textquotesingle Gowan},
\emph{Gowen},
\emph{Gowan},
\emph{Gow},
\emph{Goan}.
\end{quotation}

At present time, when historical researchers try to trace people back
using historical records, they would encounter this problem of
name variations.

Various solutions have been created to find
matching different names that refer to the same person. However,
for our extent knowledge, there is yet no public system which encodes
those solutions together and provides service of name matching.
We then decided to create one system to archieve this.

\pagebreak

\section{Research Questions}
\label{sec:rq}

From the motivation, we address our research questions as follow.

\begin{enumerate}
  \item Can we provide a web service to match names, where matching can be
    a complicated process because of the way people record their names.
  \item Can the web service act as a platform system for general names or words
    matching system so that it can be extended to other languages as well.
\end{enumerate}

The first question derives directly from the motivation.
The second question is an enhancement for the system. It can be designed
as a more general purpose matching system rather than just specified
only for Irish names. Therefore it should be extensible for any further
matching methods to be developed in the future.

\section{Objective and Aims}

The objective of this project is to provide a web service that
encodes several of matching methods and produces matching
results between two list of names.

\graffito{Matching methods implemented at the initial state of this project
are Levenshtein Distance, Soundex, Irish Soundex, and Lookup Table.}

The project aims to be a part of a bigger system, such as
genealogy research. These client systems, at some point,
they might need a service of a name matching on demand, so then they can use this
web service, providing their lists of name, methods be be used,
and threshold as inputs, and get matching results for their further usage.

We would start by focusing on Irish \textit{surname} first.
For any further kind of names we would leave it for future works.

\section{Report Structure}

This report is separated into three parts, The Background, The Solution,
and Appendix.

\begin{description}
\item[The Background:]
  Current part, states about background of this project. Introduces
  the initial problem, also some historical situations and terms
  which are not resident to computer science.
\item[The Solution:]
  The implementation to solve the problem. Details about algorithm,
  tools, language, frameworks, etc. which being used in the project.
\item[Appendix:]
  The `user manual' of the project. Presents technical aspects,
  for example, how to use the web service, or how to create an environment
  to host this project.
\end{description}
