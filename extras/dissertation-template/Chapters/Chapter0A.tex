\chapter{How to deploy the system on Ubuntu server}

Here is the list of steps we performed to deploy the system on one rental
VPS\footnote{Virtual private server \cite[]{vps}} from Digital Ocean\footnote{https://www.digitalocean.com}.
The sample machine was Ubuntu 14.04 x64, but these setup steps should work
on any recent Ubutu build as well.

This machine was setup from scratch, from creating a deployment user, up to
installing PostgreSQL, Ruby, and Ruby on Rails. We\textquotesingle re using the following
machine specifications.

\begin{itemize}
  \item Ubuntu 14.04 x64
  \item 1 core processor
  \item 512MB Ram
  \item 20GB SSD Disk
\end{itemize}

Most of these steps and wordings are taken directly from Digital Ocean\textquotesingle s
tutorials \cite[]{ubuntu} \cite[]{ubuntu2}. The setup steps are as follow.

\section{Root Login}

From your local machine, use \texttt{ssh} to connect to the remote server.
As for sample code from now, we will use \texttt{local\$} when referring
to running a code from local machine, and \texttt{remote\$} when referring
to running a code from remote machine.

\begin{minipage}{\linewidth}
\begin{lstlisting}[language=bash]
local$ ssh root@SERVER_IP_ADDRESS
\end{lstlisting}
\end{minipage}

Substitute \texttt{SERVER\_IP\_ADDRESS} with your IP address or hostname.

\section{Create a new user}

Root access is not recommend since it has very powerful privileges.
We will create a new user for deployment and other day-to-day work.

\begin{minipage}{\linewidth}
\begin{lstlisting}[language=bash]
remote$ adduser demo
\end{lstlisting}
\end{minipage}

Substitute \texttt{demo} with your prefer new user name.
You will be asked a few questions, starting with the account password
and fill in any of the additional information if you would like.

\section{Root Privileges}

Now, we have a new user account with regular account privileges.
However, we may sometimes need to do administrative tasks.

To add these privileges to our new user, we need to add the new user
to the \texttt{sudo} group. By default, on Ubuntu 14.04,
users who belong to the \texttt{sudo} group are allowed to use the sudo command.

As root, run this command to add your new user to the sudo group
(substitute \texttt{demo} with your new user).

\begin{minipage}{\linewidth}
\begin{lstlisting}[language=bash]
remote$ gpasswd -a demo sudo
\end{lstlisting}
\end{minipage}

Now you can log out and log in again with your newly created user.

\begin{minipage}{\linewidth}
\begin{lstlisting}[language=bash]
local$ ssh demo@SERVER_IP_ADDRESS
\end{lstlisting}
\end{minipage}

\section{Install rbenv}

We will install \texttt{rbenv}, which we will use to install and manage our
Ruby installation. First, update \texttt{apt-get}.

\begin{minipage}{\linewidth}
\begin{lstlisting}[language=bash]
remote$ sudo apt-get update
\end{lstlisting}
\end{minipage}

Then install the \texttt{rbenv} and Ruby dependencies with \texttt{apt-get}.

\begin{minipage}{\linewidth}
\begin{lstlisting}[language=bash]
remote$ sudo apt-get install git-core curl zlib1g-dev build-essential libssl-dev libreadline-dev libyaml-dev libsqlite3-dev sqlite3 libxml2-dev libxslt1-dev libcurl4-openssl-dev python-software-properties libffi-dev
\end{lstlisting}
\end{minipage}

Now we are ready to install \texttt{rbenv}. The easiest way to do that is to run
these commands.

\begin{minipage}{\linewidth}
\begin{lstlisting}[language=bash]
remote$ cd
remote$ git clone git://github.com/sstephenson/rbenv.git .rbenv
remote$ echo 'export PATH="$HOME/.rbenv/bin:$PATH"' >> ~/.bash_profile
remote$ echo 'eval "$(rbenv init -)"' >> ~/.bash_profile
remote$ exec $SHELL

remote$ git clone git://github.com/sstephenson/ruby-build.git ~/.rbenv/plugins/ruby-build
remote$ echo 'export PATH="$HOME/.rbenv/plugins/ruby-build/bin:$PATH"' >> ~/.bash_profile
remote$ exec $SHELL
\end{lstlisting}
\end{minipage}

This installs \texttt{rbenv} into your home directory, and sets the appropriate
environment variables that will allow \texttt{rbenv} to the active version of Ruby.

Next is to use \texttt{rbenv} to install Ruby.

\section{Install Ruby}

We will install the latest version of this time, Ruby 2.2.1.

\begin{minipage}{\linewidth}
\begin{lstlisting}[language=bash]
remote$ rbenv install -v 2.2.1
remote$ rbenv global 2.2.1
\end{lstlisting}
\end{minipage}

The \texttt{global} sub-command sets the default version of Ruby
that all of your shells will use.

You will also want to install the \texttt{bundler} gem,
to manage your application dependencies.

\begin{minipage}{\linewidth}
\begin{lstlisting}[language=bash]
remote$ gem install bundler
\end{lstlisting}
\end{minipage}

Now that Ruby is installed, next is to install Rails.

\section{Install Rails}

Install Rails 4.2.0 with this command.

\begin{minipage}{\linewidth}
\begin{lstlisting}[language=bash]
remote$ gem install rails -v 4.2.0
\end{lstlisting}
\end{minipage}

\section{Install Javascript Runtime}

A few Rails features, such as the Asset Pipeline,
depend on a Javascript runtime. We will install Node.js
to provide this functionality.

Add the Node.js PPA to \texttt{apt-get}, then update \texttt{apt-get}
and install the Node.js package.

\begin{minipage}{\linewidth}
\begin{lstlisting}[language=bash]
remote$ sudo add-apt-repository ppa:chris-lea/node.js
remote$ sudo apt-get update
remote$ sudo apt-get install nodejs
\end{lstlisting}
\end{minipage}

\section{Install PostgreSQL}

Install PostgreSQL and its development libraries.

\begin{minipage}{\linewidth}
\begin{lstlisting}[language=bash]
remote$ sudo apt-get install postgresql postgresql-contrib libpq-dev
\end{lstlisting}
\end{minipage}

\section{Create Database User}

Create a PostgreSQL superuser user with this command
(substitute the \texttt{pguser} with your own username).

\begin{minipage}{\linewidth}
\begin{lstlisting}[language=bash]
remote$ sudo -u postgres createuser -s pguser
\end{lstlisting}
\end{minipage}

\section{Get the system code}

The system code is stored at \url{git.cs.nuim.ie/repos/desem/dmsc1407},
we will get the code by using \texttt{git}.

\begin{minipage}{\linewidth}
\begin{lstlisting}[language=bash]
remote$ cd
remote$ git clone ssh://git.cs.nuim.ie/repos/desem/dmsc1407
remote$ cd dmsc1407
\end{lstlisting}
\end{minipage}

\section{Configure Database Connection}

Open your application\textquotesingle s database configuration file.

\begin{minipage}{\linewidth}
\begin{lstlisting}[language=bash]
remote$ vim config/database.yml
\end{lstlisting}
\end{minipage}

Under the default section, find the line that says \texttt{pool: 5}
and add the following lines under it. It should look something like this
(replace the \texttt{pguser} and \texttt{pguser\_password}
parts with your PostgreSQL user and password):

\begin{minipage}{\linewidth}
  \begin{lstlisting}[language=ruby, caption={config/database.yml}]
host: localhost
username: pguser
password: pguser_password
\end{lstlisting}
\end{minipage}

Save and exit.

\section{Create Application Databases}

Create your application\textquotesingle s development and test databases by using this \texttt{rake} command.

\begin{minipage}{\linewidth}
\begin{lstlisting}[language=bash]
remote$ rake db:create
\end{lstlisting}
\end{minipage}

\section{Install Puma}

Puma is an application server that enables your Rails application
to process requests concurrently.

An easy way to do this is to add it to your application\textquotesingle s \texttt{Gemfile}.

\begin{minipage}{\linewidth}
\begin{lstlisting}[language=bash]
remote$ vim Gemfile
\end{lstlisting}
\end{minipage}

At the end of the file, add the Puma gem with this line.

\begin{minipage}{\linewidth}
  \begin{lstlisting}[language=ruby, caption={Gemfile}]
gem 'puma'
\end{lstlisting}
\end{minipage}

Save and exit. To install Puma, and any outstanding dependencies, run Bundler.

\begin{minipage}{\linewidth}
\begin{lstlisting}[language=bash]
remote$ bundle
\end{lstlisting}
\end{minipage}

Puma is now installed, but we need to configure it.

\section{Configure Puma}

Before configuring Puma, you should look up the number of CPU cores
your server has. You can easily to that with this command.

\begin{minipage}{\linewidth}
\begin{lstlisting}[language=bash]
remote$ grep -c processor /proc/cpuinfo
\end{lstlisting}
\end{minipage}

Now, let\textquotesingle s add our Puma configuration to \texttt{config/puma.rb}.

\begin{minipage}{\linewidth}
\begin{lstlisting}[language=bash]
remote$ vim config/puma.rb
\end{lstlisting}
\end{minipage}

Use this Puma configuration.

\begin{minipage}{\linewidth}
  \begin{lstlisting}[language=ruby, caption={config/puma.rb}]
# Change to match your CPU core count
workers 2

# Min and Max threads per worker
threads 1, 6

app_dir = File.expand_path("../..", __FILE__)
shared_dir = "#{app_dir}/shared"

# Default to production
rails_env = ENV['RAILS_ENV'] || "production"
environment rails_env

# Set up socket location
bind "unix://#{shared_dir}/sockets/puma.sock"

# Logging
stdout_redirect "#{shared_dir}/log/puma.stdout.log", "#{shared_dir}/log/puma.stderr.log", true

# Set master PID and state locations
pidfile "#{shared_dir}/pids/puma.pid"
state_path "#{shared_dir}/pids/puma.state"
activate_control_app

on_worker_boot do
  require "active_record"
  ActiveRecord::Base.connection.disconnect! rescue ActiveRecord::ConnectionNotEstablished
  ActiveRecord::Base.establish_connection(
    YAML.load_file("#{app_dir}/config/database.yml")[rails_env])
end
\end{lstlisting}
\end{minipage}

Change the number of workers to the number of CPU cores of your server.
Then save and exit.

Now create the directories that were referred to in the configuration file.

\begin{minipage}{\linewidth}
\begin{lstlisting}[language=bash]
remote$ mkdir -p shared/pids shared/sockets shared/log
\end{lstlisting}
\end{minipage}

\section{Create Puma Upstart Script}

Create an Upstart init script so we can easily start and stop Puma,
and ensure that it will start on boot.
Download the Jungle Upstart tool from the Puma GitHub repository
to your home directory

\begin{minipage}{\linewidth}
\begin{lstlisting}[language=bash]
remote$ cd ~
remote$ wget https://raw.githubusercontent.com/puma/puma/master/
  tools/jungle/upstart/puma-manager.conf
remote$ wget https://raw.githubusercontent.com/puma/puma/master/
  tools/jungle/upstart/puma.conf
\end{lstlisting}
\end{minipage}

Now open the provided \texttt{puma.conf} file,
so we can configure the Puma deployment user.

\begin{minipage}{\linewidth}
\begin{lstlisting}[language=bash]
remote$ vim puma.conf
\end{lstlisting}
\end{minipage}

Look for the two lines that specify setuid and setgid, and replace \texttt{apps}
with the name of your deployment user and group. For example,
if your deployment user is called \texttt{demo}, the lines should look like this.

\begin{minipage}{\linewidth}
  \begin{lstlisting}[language=bash, caption={puma.conf}]
setuid demo
setgid demo
\end{lstlisting}
\end{minipage}

Save and exit.
Now copy the scripts to the Upstart services directory:

\begin{minipage}{\linewidth}
\begin{lstlisting}[language=bash]
remote$ sudo cp puma.conf puma-manager.conf /etc/init
\end{lstlisting}
\end{minipage}

The \texttt{puma-manager.conf} script references \texttt{/etc/puma.conf}
for the applications that it should manage.
Let's create and edit that inventory file now.

\begin{minipage}{\linewidth}
\begin{lstlisting}[language=bash]
remote$ sudo vim /etc/puma.conf
\end{lstlisting}
\end{minipage}

Each line in this file should be the path to an application that you
want \texttt{puma-manager} to manage.
Add the path to your application now.

\begin{minipage}{\linewidth}
  \begin{lstlisting}[language=bash, caption={/etc/puma.conf}]
/home/demo/dmsc1407
\end{lstlisting}
\end{minipage}

Save and exit. Now your application is configured to start at boot time,
through Upstart. This means that your application will start even after
your server is rebooted.

To start all of your managed Puma apps now, run this command.

\begin{minipage}{\linewidth}
\begin{lstlisting}[language=bash]
remote$ sudo start puma-manager
\end{lstlisting}
\end{minipage}

Now your Rails application\textquotesingle s production environment is running under Puma,
and it\textquotesingle s listening on the \texttt{shared/sockets/puma.sock} socket.
Before your application will be accessible to an outside user,
you must set up the Nginx reverse proxy.

\section{Install and Configure Nginx}

As Puma is not designed to be accessed by users directly,
we will use Nginx as a reverse proxy
that will buffer requests and responses between
users and your Rails application.

Install Nginx using apt-get.

\begin{minipage}{\linewidth}
\begin{lstlisting}[language=bash]
remote$ sudo apt-get install nginx
\end{lstlisting}
\end{minipage}

Now open the default server block.

\begin{minipage}{\linewidth}
\begin{lstlisting}[language=bash]
remote$ sudo vim /etc/nginx/sites-available/default
\end{lstlisting}
\end{minipage}

Replace the contents of the file with the following code block.
Be sure to replace the the \texttt{demo} and \texttt{dmsc1407} parts
with the appropriate username and application name.

\begin{minipage}{\linewidth}
  \begin{lstlisting}[language=bash, caption={/etc/nginx/sites-available/default}]
upstream app {
  # Path to Puma SOCK file, as defined previously
  server unix:/home/demo/dmsc1407/shared/sockets/puma.sock fail_timeout=0;
}

server {
  listen 80;
  server_name localhost;

  root /home/demo/dmsc1407/public;

  try_files $uri/index.html $uri @app;

  location @app {
    proxy_pass http://app;
    proxy_set_header X-Forwarded-For $proxy_add_x_forwarded_for;
    proxy_set_header Host $http_host;
    proxy_redirect off;
  }

  error_page 500 502 503 504 /500.html;
  client_max_body_size 4G;
  keepalive_timeout 10;
}
\end{lstlisting}
\end{minipage}

Save and exit. This configures Nginx as a reverse proxy, so HTTP requests
get forwarded to the Puma application server via a Unix socket.

Restart Nginx to put the changes into effect.

\begin{minipage}{\linewidth}
\begin{lstlisting}[language=bash]
remote$ sudo service nginx restart
\end{lstlisting}
\end{minipage}

\section{Finish}

You have deployed the production environment of your Ruby on Rails application using Nginx and Puma.
Now the your Rails application is accessible via your server\textquotesingle s public IP address
or domain name.
