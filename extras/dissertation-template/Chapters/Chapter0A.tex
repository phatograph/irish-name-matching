\chapter{How to deploy the system on Ubuntu server}

Here is the list of steps we performed to deploy the system on one rental
VPS\footnote{Virtual private server \cite[]{vps}} from Digital Ocean\footnote{https://www.digitalocean.com}.
The sample machine was Ubuntu 14.04 x64, but these setup steps should work
on any recent Ubutu build as well.

This machine was setup from scratch, from creating a deployment user, up to
installing PostgreSQL, Ruby, and Ruby on Rails. We\textquotesingle re using the following
machine specifications.

\begin{itemize}
  \item Ubuntu 14.04 x64
  \item 1 core processor
  \item 512MB Ram
  \item 20GB SSD Disk
\end{itemize}

Most of these steps and wordings are taken directly from Digital Ocean\textquotesingle s
tutorials \cite[]{ubuntu} \cite[]{ubuntu2}. The setup steps are as follow.

\section{Root Login}

From your local machine, use \texttt{ssh} to connect to the remote server.
As for sample code from now, we will use \texttt{local\$} when referring
to running a code from local machine, and \texttt{remote\$} when referring
to running a code from remote machine.

\begin{minipage}{\linewidth}
\begin{lstlisting}[language=bash]
local$ ssh root@SERVER_IP_ADDRESS
\end{lstlisting}
\end{minipage}

Substitute \texttt{SERVER\_IP\_ADDRESS} with your IP address or hostname.

\section{Create a new user}

Root access is not recommend since it has very powerful privileges.
We will create a new user for deployment and other day-to-day work.

\begin{minipage}{\linewidth}
\begin{lstlisting}[language=bash]
remote$ adduser demo
\end{lstlisting}
\end{minipage}

Substitute \texttt{demo} with your prefer new user name.
You will be asked a few questions, starting with the account password
and fill in any of the additional information if you would like.

\section{Root Privileges}

Now, we have a new user account with regular account privileges.
However, we may sometimes need to do administrative tasks.

To add these privileges to our new user, we need to add the new user
to the \texttt{sudo} group. By default, on Ubuntu 14.04,
users who belong to the \texttt{sudo} group are allowed to use the sudo command.

As root, run this command to add your new user to the sudo group
(substitute \texttt{demo} with your new user).

\begin{minipage}{\linewidth}
\begin{lstlisting}[language=bash]
remote$ gpasswd -a demo sudo
\end{lstlisting}
\end{minipage}

Now you can log out and log in again with your newly created user.

\begin{minipage}{\linewidth}
\begin{lstlisting}[language=bash]
local$ ssh demo@SERVER_IP_ADDRESS
\end{lstlisting}
\end{minipage}

\section{Install rbenv}

We will install \texttt{rbenv}, which we will use to install and manage our
Ruby installation. First, update \texttt{apt-get}.

\begin{minipage}{\linewidth}
\begin{lstlisting}[language=bash]
remote$ sudo apt-get update
\end{lstlisting}
\end{minipage}

Then install the \texttt{rbenv} and Ruby dependencies with \texttt{apt-get}.

\begin{minipage}{\linewidth}
\begin{lstlisting}[language=bash]
remote$ sudo apt-get install git-core curl zlib1g-dev build-essential libssl-dev libreadline-dev libyaml-dev libsqlite3-dev sqlite3 libxml2-dev libxslt1-dev libcurl4-openssl-dev python-software-properties libffi-dev
\end{lstlisting}
\end{minipage}

Now we are ready to install \texttt{rbenv}. The easiest way to do that is to run
these commands.

\begin{minipage}{\linewidth}
\begin{lstlisting}[language=bash]
remote$ cd
remote$ git clone git://github.com/sstephenson/rbenv.git .rbenv
remote$ echo 'export PATH="$HOME/.rbenv/bin:$PATH"' >> ~/.bash_profile
remote$ echo 'eval "$(rbenv init -)"' >> ~/.bash_profile
remote$ exec $SHELL

remote$ git clone git://github.com/sstephenson/ruby-build.git ~/.rbenv/plugins/ruby-build
remote$ echo 'export PATH="$HOME/.rbenv/plugins/ruby-build/bin:$PATH"' >> ~/.bash_profile
remote$ exec $SHELL
\end{lstlisting}
\end{minipage}

This installs \texttt{rbenv} into your home directory, and sets the appropriate
environment variables that will allow \texttt{rbenv} to the active version of Ruby.

Next is to use \texttt{rbenv} to install Ruby.

\section{Install Ruby}

We will install the latest version of this time, Ruby 2.2.1.

\begin{minipage}{\linewidth}
\begin{lstlisting}[language=bash]
remote$ rbenv install -v 2.2.1
remote$ rbenv global 2.2.1
\end{lstlisting}
\end{minipage}

The \texttt{global} sub-command sets the default version of Ruby
that all of your shells will use.

You will also want to install the \texttt{bundler} gem,
to manage your application dependencies.

\begin{minipage}{\linewidth}
\begin{lstlisting}[language=bash]
remote$ gem install bundler
\end{lstlisting}
\end{minipage}

Now that Ruby is installed, next is to install Rails.

\section{Install Rails}

Install Rails 4.2.0 with this command.

\begin{minipage}{\linewidth}
\begin{lstlisting}[language=bash]
remote$ gem install rails -v 4.2.0
\end{lstlisting}
\end{minipage}

\section{Install Javascript Runtime}

A few Rails features, such as the Asset Pipeline,
depend on a Javascript runtime. We will install Node.js
to provide this functionality.

Add the Node.js PPA to \texttt{apt-get}, then update \texttt{apt-get}
and install the Node.js package.

\begin{minipage}{\linewidth}
\begin{lstlisting}[language=bash]
remote$ sudo add-apt-repository ppa:chris-lea/node.js
remote$ sudo apt-get update
remote$ sudo apt-get install nodejs
\end{lstlisting}
\end{minipage}

\section{Install PostgreSQL}

Install PostgreSQL and its development libraries.

\begin{minipage}{\linewidth}
\begin{lstlisting}[language=bash]
remote$ sudo apt-get install postgresql postgresql-contrib libpq-dev
\end{lstlisting}
\end{minipage}

\section{Create Database User}

Create a PostgreSQL superuser user with this command
(substitute the \texttt{pguser} with your own username).

\begin{minipage}{\linewidth}
\begin{lstlisting}[language=bash]
remote$ sudo -u postgres createuser -s pguser
\end{lstlisting}
\end{minipage}

\section{Get the system code}

The system code is stored at \url{git.cs.nuim.ie/repos/desem/dmsc1407},
we will get the code by using \texttt{git}.

\begin{minipage}{\linewidth}
\begin{lstlisting}[language=bash]
remote$ cd
remote$ git clone ssh://git.cs.nuim.ie/repos/desem/dmsc1407
remote$ cd dmsc1407
\end{lstlisting}
\end{minipage}

\section{Configure Database Connection}
