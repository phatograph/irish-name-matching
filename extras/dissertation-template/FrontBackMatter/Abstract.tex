% Abstract

\pdfbookmark[1]{Abstract}{Abstract} % Bookmark name visible in a PDF viewer

\begingroup
\let\clearpage\relax
\let\cleardoublepage\relax
\let\cleardoublepage\relax

\chapter*{Abstract}

Before the first Irish civil registration on 1864, census meterials
were mostly lost or incomplete. So genealogical researches use
parish records and also some `census substitute' documents,
such as land ownership and tenancy records. However, some of these documents
do not contain enough information in identify individuals.
Some of them contains name and address, whereas others might contain only name.

\emph{Record linkage} is one method to gather scattered information among many documents.
It uses a person\textquotesingle s name as a reference to link that
person\textquotesingle s information between many documents. With patience,
a more complete information about that person can be obtained.

% Apparently linking or matching person\textquotesingle s name is important in the process.
% Unfortunately, in the 19\textsuperscript{th} century, in Ireland, there was no standard
% of the spelling of names, handwriting could be difficult to read
% and contractions or abbreviations were often used. Many people were not literate,
% so they asked literate people to write their names.
% This way even names with the same pronounciation and for the same individual
% could be written in many different ways, depending on who wrote them.
%
% In addition to the various ways of spelling one\textquotesingle s name,
% people from this time also often use Irish names which equivalent to modern names,
% for example, Irish version of `Smith' could be `Gowan'.
% There are also some Irish prefixes like `O\textquotesingle', `M\textquotesingle', `Mac',
% etc. When combined together this would result in `O\textquotesingle Gowan' or
% `M\textquotesingle Gowen', and so on.
%
% At present time, when historical researchers try to trace people back
% using historical records, they would encounter this problem of
% name variations.
%
% Various solutions have been created to find
% matching different names that refer to the same person. However,
% for our extent knowledge, there is yet no public system which encodes
% those solutions together and provides a service of name matching.
% This project is to create one system to archieve this.
%
% We successfully developed an extensible web service system to match names.
% The system is initially encoded with 4 matching algorithms,
% Levenshtein distance, soundex, Irish soundex, and lookup table.
% We also present a web interface for a client to use the system
% from the web browser.
%
% The system is designed to be extended with simple inheritance, thus developer
% can understand and develop further algorithm easily. In early state
% simple design is enough to serve the purpose, so we follow the
% \emph{Kiss principle}.

\endgroup
\vfill
