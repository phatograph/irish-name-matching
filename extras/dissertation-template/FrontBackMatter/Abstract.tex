% Abstract

\pdfbookmark[1]{Abstract}{Abstract} % Bookmark name visible in a PDF viewer

\begingroup
\let\clearpage\relax
\let\cleardoublepage\relax
\let\cleardoublepage\relax

\chapter*{Abstract}

Before the first Irish civil registration on 1864, census materials
were mostly lost or incomplete. So genealogical research uses
parish records and also some `census substitute' documents,
such as land ownership and tenancy records. However, some of these documents
may not contain enough information in identify individuals.
Some of them contains a name and address,
whereas others might contain only a name.

\emph{Record linkage} is one method to gather scattered information among many documents.
It uses a person\textquotesingle s name as a reference to link that
person\textquotesingle s information between many documents. With patience,
a more complete information about that person can be obtained.

Therefore linking or matching a person\textquotesingle s name is important in the process.
Unfortunately, in the 19\textsuperscript{th} century, in Ireland, there was no standard
spelling of names, handwriting could be difficult to read
and contractions or abbreviations were often used. The names with the same
pronunciation and for the same individual could be written in many different ways.
Moreover, names in the Irish language which are equivalent to English names
were used, for example, Irish version of `Smith' could be `Gowan'.
A further complication is that historical  and genealogical research often
requires large quantities of names to be matched.

To handle these name variations, various solutions have been created to find
matching different names that refer to the same person. However,
for our extent knowledge, there is yet no public system which encodes
those solutions together and provides a service of bulk name matching.
Thus, we developed a web service system using Ruby on Rails framework
to achieve our goal. The system is initially encoded with 4 matching algorithms,
Levenshtein distance, soundex, Irish soundex, and lookup table.
We also present a web interface for a client to use the system
from the web browser. It is designed to be simple and extensible from using
inheritance.

The system performs matchings on large quantities of names in a reasonable time.
We test our system with 12,944 name matchings and the result were completed
in no more than half a minute (28,786 milliseconds, to be precise).
However, the system consumes a large amount of memory (around 373 megabytes).
We believe that, with proper optimisation, we would reduce the memory usage
along with a shortened processing time. Further matching algorithms could also
be implemented for names in other languages, so that it can handle
a broader domain of names.

\endgroup
\vfill
